\newpage

\section*{Abstrakt}

\begin{tabular}{ll}
    \textbf{Autor:}                       & Jakub Jursa \\
    \textbf{Názov diplomovej práce:}      & Dátové štruktúry pre uchovávanie
    sekvenovacích dát \\
    \textbf{Škola:}                       & Univerzita Komenského v
                                            Bratislave \\
    \textbf{Fakulta:}                     & Fakulta matematiky, fyziky a
                                            informatiky \\
    \textbf{Katedra:}                     & Katedra informatiky \\
    \textbf{Vedúci diplomovej práce:}     & Mgr. Tomáš Vinař, PhD. \\
    \textbf{Konzultant diplomovej práce:} & Mgr. Vladimír Boža \\
    \textbf{Rozsah práce:}                & \pageref{LastPage} strán \\
    \\
    Bratislava, máj \the\year             & {} \\
    \\
    \multicolumn{2}{p{15.3cm}}{
        V práci sa zaoberáme návrhom a implementáciou dátovej štruktúry pre efektívne indexovanie krátkych reťazcov pochádzajúcich zo sekvenovacích dát (\emph{sequencing reads}) s možnosťou efektívneho vyhľadávania tých readov, ktoré obsahujú daný podreťazec pevne danej dĺžky. Podstatou tejto dátovej štruktúry je komprimácia vstupnej sady readov za použitia algoritmov na zostavovanie genómu (pomocou vhodného softvérového nástroja) a následná efektívna indexácia cez FM-index. Výsledná implementácia je pamäťovo výrazne efektívnejšia než existujúce riešenia pri zachovaní podobnej rýchlosti odpovedania na dotazy.
    }\\
    \\        
    \textbf{Kľúčové slová:}               & dátové štruktúry a algoritmy, sekvenovacie ready, \\
                                          & indexovanie, bioinformatika
\end{tabular}