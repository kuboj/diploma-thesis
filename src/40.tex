V tejto kapitole popíšeme našu implementáciu CR-indexu v jazyku \texttt{C++}, použité knižnice, a jej štruktúru a rozhrania.

\section{Použité knižnice}

Ako programovací jazyk sme zvolili \texttt{C++}, konkrétne verziu \texttt{C++11}. Zdrojový kód by mal byť (po doinštalovaní potrebných závislostí) skompilovateľný kompilátorom \texttt{gcc} verzie 4.9.1 a vyššie. Pri implementácii sme využili nasledujúce knižnice resp. programy:

\subsubsection{Boost}
\emph{Boost} je rozsiahla sada univerzálnych knižníc pre programovací jazyk \texttt{C++}. My sme z nej využili podknižnice \texttt{libboost-filesystem} a \texttt{libboost-system} na prácu so súborovým systémom a procesmi.

\subsubsection{SGA}
\emph{SGA} (\emph{String Graph Assembler}) \cite{SD11} je implementácia \emph{de novo genome} assemblera založeného na koncepte grafových reťazcov v jazyku \texttt{C++} a je dostupná na stránke \url{https://github.com/jts/sga}. Tento assembler sme využili pri implementácii CR-indexu hneď na niekoľkých miestach:

\begin{itemize}
\item V časti \ref{ssec:korekcia_readov} -- Korekcia readov, sme realizovali korekčnú funkciu $corr_R$ pomocou podprogramu SGA \texttt{sga correct}. Ten očakáva na vstupe okrem sady readov vo \texttt{fastq} formáte aj index tejto sady, ktorý sme vybudovali pomocou podprogramu \texttt{sga index}.
\item Ďalej v časti \ref{ssec:budovanie_indexu} -- Budovanie indexu sme pomocou SGA konštruovali kontigy. Najprv sme pomocou \texttt{sga index} vybudovali index readov, potom pomocou \texttt{sga overlap} našli prekryvy readov a nakoniec cez \texttt{sga assemble} skonštruovali kontigy.
\end{itemize}

\subsubsection{SDSL}
\emph{SDSL} (\emph{Succinct Data Structure Library}) [GBMP14] je sada knižníc obsahujúca implementáciu \emph{úsporných dátových štruktúr} pre programovací jazyk \texttt{C++}. Zahŕňa množstvo rôznych implementácií napríklad bitových vektorov, celočíselných vektorov, wavelet stromov, komprimovaných sufixových polí a podobne. My sme túto knižnicu využili na implementáciu FM-indexu, ktorý používame v CR-indexe hneď dvakrát -- prvýkrát v konštrukčnej fáze, keď v zreťazených kontigoch hľadáme chýbajúce ready a druhýkrát konštruujeme FM-index nad superstringom a potom ho využívame pri dotazovaní.

\section{Inštalácia}
Zdrojový kód našej implementácie je verejne dostupný na Github-e na URL \url{https://github.com/kuboj/CR-index} pod licenciou MIT. Pred vybudovaním zdrojových kódov CR-indexu je najprv potrebné mať nainštalované všetky závislosti spomínane v predchádzajúcej časti. \texttt{Boost}, \texttt{PStreams}, \texttt{libboost-filesystem} a \texttt{libboost-system} zvyknú byť široko dostupné pre Unixové operačné systémy vo forme balíčkov pre daný systém. Po naklonovaní repozitára lokálne stačí spustiť príkaz \texttt{make} pre vybudovanie modulov alebo \texttt{make examples} resp. \texttt{make benchmarks} pre vybudovanie príkladov respektíve testovacích programov.

\section{Vonkajšia štruktúra}

\subsubsection{Súborová štruktúra}

\begin{minipage}[t]{0.20\linewidth}
    \vspace{0pt}
    \texttt{benchmark/} \\
    \texttt{bin/} \\
    \texttt{build/} \\
    \texttt{examples/}  \\
    \texttt{include/}   \\
    \texttt{src/}       \\
    \texttt{tools/}     \\
    \texttt{LICENSE}     \\
    \texttt{Makefile}     \\
    \texttt{README.md}
\end{minipage}
\hspace{0.3cm}
\begin{minipage}[t]{0.8\linewidth}
    \vspace{0pt}
    obsahuje dva testy -- \texttt{construct.cpp} a \texttt{query.cpp} \\
    táto zložka po vybudovaní obsahuje spustiteľné súbory \\
    obsahuje vybudované objektové súbory \\
    v tejto zložke sa nachádzajú príklady \\
    obsahuje hlavičkové súbory (\emph{header files}) \\
    obsahuje samotné zdrojové kódy knižnice \\
    zoskupuje pomocné nástroje \\
    súbor obsahujúci MIT licenciu \\
    obsahuje pravidlá pre vybudovanie knižnice pomocou nástroja \texttt{make} \\
    stručné inštrukcie pre inštaláciu a používanie
\end{minipage}

\subsubsection{benchmark/construct.cpp}
Úlohou tohto testu je odmerať množstvo spotrebovanej pamäte (a čas potrebný pre konštrukciu) pre jednotlivé implementácie štruktúr indexujúcich ready. Jeho rozhranie príkazového riadku vyzerá nasledovne:

\bigskip
\begin{pseudocode}[caption={Rozhranie príkazového riadku test construct.cpp}]
$ construct --help
Construct index 
Usage: 
  construct <index_type> <filename> <read_length> <query_length>
index_type: cr, hash, gk
Example: 
  ./bin/construct cr bacteria.fastq 100 13
\end{pseudocode}
\bigskip

Na vstupe očakáva typ indexu, ktorým bude indexovať vstupnú sadu readov -- \texttt{cr} označuje našu implementáciu CR-indexu, \texttt{hash} implementáciu pomocou hash mapy (mierne vylepšenú oproti časti \ref{sec:riesenie_s_pouzitim_hash_mapy}) a \texttt{gk} označuje konštrukciu indexu pomocou knižnice GkArray (ktorá je popísaná v časti \ref{sec:riesenie_s_pouzitim_gkarray}). Parameter \texttt{filename} predstavuje cestu k súboru so sadou readov vo formáte FASTQ, parameter \texttt{read\_length} označuje dĺžku readu a posledný parameter, \texttt{query\_length} dĺžku dotazu (ktorú sme v predchádzajúcom texte označovali ako $k$).

Ukážkový výstup tohto testu je nasledovný:

\bigskip
\begin{pseudocode}[caption={Ukážkový výstup testu construct.cpp}]
$ ./construct cr ../genomes/bacteria1M_10E.fastq 100 13                                                       
index_type: cr
read_filename: ../genomes/bacteria1M_10E.fastq
read_length: 100
query_length: 13
Total reads size: 100000000
Superstring size:  9456094
Compress ratio: 10.575191
Construction took 252.935495s
Referenced memory: 23536kB
OK.
\end{pseudocode}
\bigskip

Meranie spotrebovaného množstva pamäte prebieha na konci konštrukcie -- snahou je teda zmerať, koľko pamäte zaberá samotný index a nie maximum spotrebovanej pamäte počas konštrukcie\footnote{Aj keď aj to je často sledovaný parameter.}. Toto meranie je (momentálne) implementované len pre Unixové operačné systémy a je realizované pomocou analýzy mapovania pamäti procesu, ktoré čítame zo súboru \texttt{/proc/<pid>/smaps}.

\subsubsection{benchmark/query.cpp}
Tento najprv skonštruuje daný typ indexu a potom meria čas potrebný na odpovedanie na všetky dotazy zo vstupného súboru. Rozhranie príkazového riadku je nasledovné:

\bigskip
\begin{pseudocode}[caption={Rozhranie príkazového riadku test query.cpp}]
$ ./query 
Construct index 
Usage: ./query <index_type> <reads_filename> <queries_filename>\
  <read_length> <query_length> <query_type>
index_type: cr, hash, gk
query_type: index, read
Example: 
  ./query cr bacteria.fastq queries.data 101 13 index
\end{pseudocode}
\bigskip

Parametre \texttt{index\_type}, \texttt{reads\_filename}, \texttt{read\_length}, \texttt{query\_length} sú rovnaké ako v predchádzajúcom teste. Parameter \texttt{queries\_filename} označuje cestu ku súboru obsahujúcemu sadu dotazov, každý dĺžky \texttt{query\_length}, oddelených znakom nového riadku. Parameter \texttt{query\_type} obsahuje typ dotazu, ktorý vykonávame -- v prípade, že sa pýtame len na indexy readov, ktoré danú vzorku obsahujú je tento rovný \emph{index}, ak požadujeme pri dotaze vrátenie celých readov ako reťazcov, tak tento parameter musí mať hodnotu \texttt{read}.

Ukážkový výstup tohto testu je nasledovný:

\bigskip
\begin{pseudocode}[caption={Ukážkový výstup testu query.cpp}]
$ ./query cr ../genomes/bacteria1M_10E.fastq queries 100 13 index
index_type: cr
read_filename: ../genomes/bacteria1M_10E.fastq
queries_filename: queries
read_length: 100
query_length: 13
query_type: index
number of queries: 10000
Constructing index ... 
Total reads size: 100000000
Superstring size:  9456094
Compress ratio: 10.575191
Querying ...
Querying took 2.399451s
OK.
\end{pseudocode}
\bigskip

\section{Vnútorná štruktúra}
Samotné zdrojové kódy CR-indexu sa nachádzajú v zložkách \texttt{include/} (hlavičkové súbory) a \texttt{src/} (súbory implementácie). Zložka \texttt{src/} obsahuje súbory \texttt{util.cpp} (čo je súbor pomocných funkcií na prácu s reťazcami, spúšťanie podprocesov a podobne), \texttt{hash\_index.cpp} (predstavuje implementáciu pomocou hash mapy), \texttt{fm\_wrapper.cpp} (trieda vhodne zapuzdrujúca implementáciu FM-indexu v knižnici SDSL) a súbor \texttt{cr\_index.cpp}, ktorý implementuje triedu CRIndex.

\subsubsection{Trieda CRIndex}
Verejné rozhranie tejto triedy vyzerá nasledovne:

\bigskip
\begin{pseudocode}[language=c++,caption={Verejné rozhranie triedy CRIndex}]
class CRIndex {
  public:
    static const bool DEFAULT_VERBOSITY;
    static const int DEFAULT_READ_LENGTH;
    static bool verbose;

    CRIndex(string path, int read_length = DEFAULT_READ_LENGTH,
                bool verbose = DEFAULT_VERBOSITY);
    CRIndex(string superstring, vector<t_pos> positions,
                vector<t_diff> diff,
                int read_length = DEFAULT_READ_LENGTH,
                bool verbose = DEFAULT_VERBOSITY);
    vector<int> find_indexes(const string& s);
    vector<string> find_reads(const string& s);
    ~CRIndex();

    static tuple<string, vector<t_pos>, vector<t_diff>>
                     preprocess(string path,
                         bool verbose = DEFAULT_VERBOSITY);
    ...                     
\end{pseudocode}
\bigskip

Typy \texttt{t\_pos} a \texttt{t\_diff} označujú typ prvkov polí \texttt{pos} resp. \texttt{diff} (a sú definované v príslušnom hlavičkovom súbore).

Prvý konštruktor očakáva na vstupe reťazec s cestou ku súboru so sadou readov vo formáte FASTQ, dĺžku readu a logickú premennú, ktorá zapína obsiahlejšie výpisy. Tento konštruktor kompletne vykoná konštrukciu celého CR-indexu.

Statická metóda \texttt{preprocess} spracuje vstupnú sadu readov a vráti superstring a polia \texttt{pos} a \texttt{diff}. Túto metódu je možné použiť s druhým konštruktorom -- ak by sme potrebovali tieto medzivýsledky uložiť alebo nejak analyzovať. 

Dotazy sa vykonávajú pomocou metód \texttt{find\_reads} a \texttt{find\_indexes}, ktoré pre reťazec na vstupe vrátia ready, resp. indexy readov, ktoré túto vzorku obsahujú\footnote{Alebo jej reverzný komplement.}.

Minimalistický príklad použitia by teda vyzeral nasledovne:

\bigskip
\begin{pseudocode}[language=c++,caption={Príklad použitia triedy CRIndex}]
...
#include "cr_index.hpp"

CRIndex cr = CRIndex("/tmp/staphylococcus.fastq", 101);
for (auto s : cr.find_reads("ACTGGGTCCACCCA")) {
  cout << s << endl;
}
\end{pseudocode}
\bigskip